\section{Preprocessing dei dati}

In questa sezione vengono applicate trasformazioni ai dati al fine di aumentare la qualità dell'output dei vari algoritmi e di garantire delle performance migliori.

Non si è ritenuto necessario aggregare o eliminare attributi, in quanto il numero di questi è molto ridotto e non sono presenti attributi linearmente dipendenti tra di loro.

\subsection{Importazione in Weka}

Tutte le analisi sui dati sono state effettuate tramite il software Weka\footnote{\url{https://www.cs.waikato.ac.nz/ml/weka/}}.

Il file, come detto nella \Vref{subsec:intro:dataset}, è di fatto un CSV senza linea di \emph{header}.
Purtroppo, l'importazione del file come CSV senza effettuare modifiche non ha funzionato, anche mettendo a \texttt{true} il flag \texttt{noHeaderRowPresent};
si è dunque proceduto aggiungendo manualmente al file una linea di intestazione con i nomi delle colonne, risolvendo il problema.

Ispezionando i dati importati, risulta subito evidente che Weka non ha compreso automaticamente la natura dei dati:
gli attributi che dovrebbero essere \emph{nominali} o \emph{binari}, avendo valori composti da numeri, vengono invece considerati \emph{numerici}.
Si è dunque proceduto applicando il filtro \texttt{NumericToNominal} per trasformare gli attributi 1, 6, 7 e 8 in nominali
e il filtro \texttt{NumericToBinary} per trasformare gli attributi 3, 4 e 5 in binari (nonostante Weka li tratti, di fatto, come nominali).
Poiché il filtro \texttt{NumericToBinary} rinomina gli attributi aggiungendo ``\texttt{\_binarized}'',
questi suffissi sono stati rimossi tramite filtro \texttt{RenameAttribute} passando la regex \verb|([\s\S]+)_binarized|.

\subsection{Pulizia}

In questa fase si è controllato il dataset al fine di rilevare eventuali problemi con i dati e provare a gestirli.
Fortunatamente, il controllo manuale ha confermato quanto si poteva comprendere anche dalla tabella descrittiva sul sito riportato nella~\Cref{note:dataset}:
il dataset risulta privo di record duplicati e di attributi mancanti.

Non è stato dunque necessario pulire ulteriormente il dataset oltre ai filtri applicati per la corretta importazione in Weka.

\subsection{Normalizzazione}

Nell'applicare l'algoritmo \texttt{IBk}, si sono fatte delle prove anche mantenendo gli attributi come \emph{numerici},
ritenendo che l'equidistanza con cui sono stati rappresentati dai conoscitori del dominio potesse essere rilevante,
soprattutto con un algoritmo basato sul calcolo della distanza rispetto ai vicini.
In questo caso, dunque, è stata applicata la \emph{normalizzazione} ai valori numerici, per rendere i range omogenei tra di loro.
